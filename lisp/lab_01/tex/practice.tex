\chapter{Практические задания}

\section{Представить следующие списки в виде списочные ячеек}

1. \text{'(open close halph)}

\img{25mm}{r_1}{}

2. \text{'((open1) (close2) (halph3)) )}

\img{42mm}{r_2}{}

3. \text{'((one) for all (and (me (for you)))) }

\img{56mm}{r_3}{}

\newpage

4. \text{'((TOOL) (call))}

\img{45mm}{r_4}{}

5. \text{'((TOOL1) ((call2)) ((sell)))}

\img{55mm}{r_5}{}

6. \text{'(((TOOL) (call)) ((sell)))}

\img{55mm}{r_6}{}

\section{Используя только функции CAR и CDR, написать выражения,
возвращающие i-ый элемент списка}

1. второй
\begin{lstlisting}
    (CAR (CDR '(1 2 3)))        ; 2
    (CADR     '(1 2 3) )        ; 2
\end{lstlisting}

2. третий
\begin{lstlisting}
    (CAR (CDR (CDR '(1 2 3))))  ; 3
    (CADDR         '(1 2 3)  )  ; 3
\end{lstlisting}

3. четвертый
\begin{lstlisting}
    (CAR (CDR (CDR (CDR '(1 2 3 4 )))))  ; 4
    (CADDR              '(1 2 3 4)    )  ; 4
\end{lstlisting}

\section{Что будет в результате вычисления выражений?}

1. \text{(CAADR '((blue cube) (red pyramid)))}
\begin{lstlisting}
     (CDR '((blue cube) (red pyramid)))      ; ((red pyramid))
     (CAR '((red pyramid)))                   ; (red pyramid)
     (CAR '(red pyramid))                     ; red

     Ответ: red
\end{lstlisting}

2. \text{(CDAR '((abc) (def) (ghi)))}
\begin{lstlisting}
     (CAR '((abc) (def) (ghi)))             ; (abc)
     (CDR '(abc))                           ; NIL

     Ответ: NIL
\end{lstlisting}


3. \text{(CADR '((abc) (def) (ghi)))}
\begin{lstlisting}
     (CDR '((abc) (def) (ghi)))             ; ((def) (ghi))
     (CAR '((def) (ghi)))                   ; (def)

     Ответ: (def)
\end{lstlisting}

\newpage

4. \text{(CADDR '((abc) (def) (ghi)))}
\begin{lstlisting}
     (CDR '((abc) (def) (ghi)))             ; ((def) (ghi))
     (CDR '((def) (ghi)))                   ; ((ghi))
     (CAR '((ghi)))                         ; (ghi)

     Ответ: (ghi)
\end{lstlisting}


\section{Напишите результат вычисления выражений и объясните как он получен}

Апостроф (quote) --- блокирует вычисление своего аргумента.

Функция list создает и возвращает список, у которого голова --- это первый аргумент, хвост --- все остальные аргументы.

Функция cons включает новый элемент в начало списка. Если вторым аргументом передан атом, а не список, то создает точечную пару. 

\begin{lstlisting}
    (list 'Fred 'and 'Wilma)          ; (Fred and Wilma)
    (list 'Fred '(and Wilma))         ; (Fred (and Wilma))
    (cons Nil Nil))                   ; (Nil)
    (cons T Nil)                      ; (T)
    (cons Nil T)                      ; (Nil . T)
    (list Nil)                        ; (Nil)
    (cons '(T) Nil)                   ; ((T))
    (list '(one two) '(free temp))    ; ((one two) (free temp))
    (cons 'Fred '(and Wilma))         ; (Fred and Wilma)
    (cons 'Fred '(Wilma))             ; (Fred Wilma)
    (list Nil Nil)                    ; (Nil Nil)
    (list T Nil)                      ; (T Nil)
    (list Nil T)                      ; (Nil T)
    (cons T (list Nil))               ; (T Nil)
    (list '(T) Nil)                   ; ((T) Nil)
    (cons '(one two) '(free temp))    ; ((one two) free temp)
\end{lstlisting}

\section{Написать лямбда-выражение и соответствующую функцию}

1. Написать функцию (f ar1 ar2 ar3 ar4), возвращающую список:
((ar1 ar2) (ar3 ar4))

\begin{lstlisting}
    (
    defun 
        f1 
        (ar1 ar2 ar3 ar4)
        (list  (list ar1 ar2) (list ar3 ar4)  )
    )
    (f1 1 2 3 4)        ; ((1 2) (3 4))
\end{lstlisting}

\begin{lstlisting}
    (
        (
        lambda 
            (ar1 ar2 ar3 ar4)
            (list  (list ar1 ar2) (list ar3 ar4)  )
        )
        1 2 3 4         
    )                   ; ((1 2) (3 4))
\end{lstlisting}

2. Написать функцию (f ar1 ar2), возвращающую ((ar1) (ar2))

\begin{lstlisting}
    (
    defun
        f2 
        (ar1 ar2)
        (list (list ar1) (list ar2))
    )
    f2 (1 2)         ; ((1) (2)) 
\end{lstlisting}

\begin{lstlisting}
    (
        (
        lambda 
            (ar1 ar2)
            (list (list ar1) (list ar2))  
        ) 1 2
    )                ; ((1) (2)) 
\end{lstlisting}


3. Написать функцию (f ar1), возвращающую (((ar1)))

\begin{lstlisting}
    (
     defun
        f3
        (ar1)
        (list (list (list ar1)))
    )
    f3(1)           ; (((1)))
\end{lstlisting}

\begin{lstlisting}
    (
        (
        lambda 
            (ar1)
            (list (list (list ar1)))
        )
        1 
    )                ; (((1)))
\end{lstlisting}

\subsection*{Представить результаты в виде списочных ячеек}

1. ((ar1 ar2) (ar3 ar4))
\img{45mm}{r_7}{}

\newpage

2. ((ar1) (ar2))
\img{50mm}{r_8}{}

3. (((ar1)))
\img{70mm}{r_9}{}