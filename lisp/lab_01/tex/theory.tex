\chapter{Теоретические вопросы}

\section{Элементы языка: определение, синтаксис, представление в памяти}

В программировании на языке Лисп используются символы и построенные из них символьные структуры. 

\textbf{Символ} --- это имя, состоящее из букв, цифр и специальных знаков, которое обозначает какой-нибудь предмет, объект, вещь, действие из реального мира. 

Примеры символов:
\begin{lstlisting}
    x
    defun
    STep-1984
\end{lstlisting}

Символы \textbf{T} и \textbf{NIL} имеют в Лиспе специальное назначение: T обозначает логическое значение истина (true), а NIL --- логическое значение ложь (false). Символом NIL также обозначается пустой список. 

Наряду с символами в Лиспе используются и \textbf{числа}, которые как и символы, записываются при помощи ограниченной пробелами последовательности знаков. 

Примеры чисел:
\begin{lstlisting}
    746
    -3.14
    3.055E8
\end{lstlisting}

Символы и числа представляют собой те простейшие объекты Лиспа, из которых строятся остальные структуры. Поэтому их называют атомарными объектами или просто \textbf{атомами}. 
$$\textbf{Атомы = символы + T + NILL + самоопределимые атомы}$$

Cамоопределимые атомы --- натуральные числа, дробные числа, вещественные числа, строки --- последовательность
символов, заключенных в двойные апострофы (например "abc")

Более сложные данные --- \textbf{списки} и \textbf{точечные пары}, которые строятся из унифицированных структур --- блоков памяти --- бинарных узлов. 

\begin{lstlisting}
    Точечная пара ::= (<атом> . <атом>) | 
                      (<атом> . <точечная пара>) | 
                      (<точечная пара> . <атом>) | 
                      (<точечная пара> . <точечная пара>) 
\end{lstlisting}

Примеры точечных пар:
\begin{lstlisting}
    (A . B)
    (A . (B . (C . NIL)))
\end{lstlisting}
\img{30mm}{t_1}{Представление в памяти точечной пары (A . B)}

% Список --- динамическая структура данных, которая может быть пустой или непустой. Если она не пустая, то состоит из двух элементов: голова (любая структура) и хвост (список).

\begin{lstlisting}
    Список ::= <пустой список> | <непустой список>, где
                
    <пусой сисок> ::= ( ) | Nil,
    <непустой список>::= (<первый элемент> . <хвост>),
    <первый элемент> ::= <S-выражение>,
    <хвост> ::= <список>
\end{lstlisting}

Примеры списков:
\begin{lstlisting}
    (A B C)
    (A (B (C NIL)))
\end{lstlisting}
\img{30mm}{t_2}{Представление в памяти списка (A  B)}

Атомы и точечные выражения называются \textbf{символьными выражениями} или \textbf{S-выражениями}.
\begin{lstlisting}
    S-выражение ::= <атом> | <точечная пара>
\end{lstlisting}

\img{60mm}{t_3}{Символьные выражения}

\subsection*{Синтаксис}

Лисп является регистронезависимым языком. Универсальным разделителем является пробел. Наличие скобок является признаком структуры --- списка или точечной пары. Любая структура заключается в круглые скобки:
\begin{lstlisting}
    (A . B)         ; точечная пара
    (A)             ; список из одного элемента
    () или Nil      ; пустой список
    (A B C D)       ; одноуровневый список
    (A (B C))       ; структурированный список
\end{lstlisting}

\section{Особенности языка Lisp. Структура программы. Символ апостроф.}

Отличительные особенности языка Лисп: все можно представить в виде функций; только символьная обработка.

В зависимости от контекста одни и те же значения могут играть роль переменных или констант.  

Символ \textbf{апостроф} --- синоним \textbf{quote}.
quote блокирует вычисление своего аргумента. В качестве значения выдает сам аргумент не вычисляя его. Интерпретатор Лиспа, считывая начинающееся с апострофа выражение, автоматически преобразует его в соответствующий вызов функции quote. 

Примеры:
\begin{lstlisting}
    '(+ 2 4)            ; (+ 2 4)
    '(a b '(c d))       ; (a b (c d))
    (quote quote)       ; quote
    'quote              ; quote
\end{lstlisting}

\section{Базис языка Lisp. Ядро языка.}

\textbf{Базис языка} --- минимальный набор инструментов и структур данных, который позволяет решать любые задачи.

\begin{lstlisting}
    Базис Lisp = атомы + структуры + 
                 базовые функции + базовые функционалы
\end{lstlisting}

\textbf{Функция} --- правило, по которому каждому значению одного или нескольких аргументов ставится в соответствие конкретное значение результата.

Примеры функций:
\begin{lstlisting}
    CAR         ; возвращает головную часть списка
    CDR         ; возвращает хвостовую часть списка
    CONS        ; включить новый элемент в начало списка
    ATOM        ; проверить, является ли аргумент атомом
    EQ          ; проверить тождественность двух символов
\end{lstlisting}

\textbf{Функционал (функция высшего порядка)} --- функция, аргументом или результатом которой является другая функция.

Примеры функционалов:
\begin{lstlisting}
    APPLY       ; применить функцию к списку аргументов
    FUNCALL     ; вызвать функцию с аргументами
\end{lstlisting} 

\textbf{Ядро} --- основные действия, которые наиболее часто используются. Такие функции системы обычно реализовываны в виде машинных подпрограмм.



