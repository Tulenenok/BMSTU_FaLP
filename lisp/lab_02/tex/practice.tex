\chapter{Практические задания}

\section{Составить диаграмму вычисления следующих выражений}

1. \text{(equal 3 (abs -3))}

\img{78mm}{s_1}{}

2. \text{(equal (+ 1 2) 3)}

\img{78mm}{s_2}{}

3. \text{(equal (*4 7) 21)}

\img{78mm}{s_3}{}

4. \text{(equal (* 2 3) (+ 7 2))}

\img{112mm}{s_4}{}

5. \text{(equal (- 7 3) (* 3 2))}

\img{95mm}{s_5}{}

6. \text{(equal (abs (- 2 4)) 3)}

\img{95mm}{s_6}{}

\newpage

\section{Написать функцию, вычисляющую гипотенузу прямоугольного треугольника по заданным катетам и составить диаграмму ее вычисления}

\begin{lstlisting}
    (
    defun
        hypotenuse
        (a, b)
        (sqrt (+ (* a a) (* b b) ) )
    )

    (hypotenuse 3 4)        ; 5
\end{lstlisting}


\img{140mm}{h_1}{}

\newpage


\section{Каковы результаты вычисления следующих выражений?}

\begin{lstlisting}
     (list 'a c)    =>    ERROR: Variable C is unbound.

     Можно предложить два решения: 
     1. (list 'a 'c)                ; сделать c символом
     2. (let ((c 3)) (list 'a c))   ; задать c некоторое значение
\end{lstlisting}

\begin{lstlisting}
     (cons 'a (b c))    =>    ERROR: Variable C is unbound.
     (cons 'a (b 'c))   =>    ERROR: Function 'B' undefined.

     Можно предложить два решения: 
     1. (cons 'a '(b c))                ; сделать (b c) списком из двух                                  символов
     2. (defun b (c) (+ c 3))           ; задать функцию b  
        (let ((c 5)) (cons 'a (b c)))   ; задать c некоторое значение
\end{lstlisting}

\begin{lstlisting}
     (cons 'a '(b c))      =>    (A B C)
\end{lstlisting}

\begin{lstlisting}
     (caddr (1 2 3 4 5))   =>   ERROR: Bad function designator `1'

     Решение: 
     1. (caddr '(1 2 3 4 5))   ; заблокировать вычисление (1 2 3 4 5)
\end{lstlisting}

\begin{lstlisting}
     (cons 'a 'b 'c)    =>   ERROR: too many arguments

     Можно предложить два решения: 
     1. (cons 'a '(b c))      ; передать вторым аргументом список
     2. (list 'a 'b 'c)       ; использовать функцию list
\end{lstlisting}

\begin{lstlisting}
     (list 'a (b c))    =>   ERROR: Variable C is unbound.

     Можно предложить два решения: 
     1. (list 'a '(b c))               ; заблокировать вычисление 
                                         второго аргумента
     2. (defun b (c) (cons c Nil))     ; определить функцию b
        (let ((c 5)) (list 'a (b c)))  ; определить аргумент c
\end{lstlisting}


\begin{lstlisting}
     (list a '(b c))    =>   ERROR: Variable A is unbound.

     Можно предложить два решения: 
     1. (list 'a '(b c))               ; заблокировать вычисление 
                                         первого аргумента
     2. (let ((a 5)) (list a '(b c)))  ; определить аргумент a
\end{lstlisting}

\begin{lstlisting}
     (list (+ 1 '(length '(1 2 3))))    =>   ERROR: Not a number!

     Решение: 
     1. (list (+ 1 (length '(1 2 3)))) ; не блокировать вычисление 
                                         второго аргумента
\end{lstlisting}




\section{Написать функцию longer-then от двух списков аргументов, которая возвращает T, если первый аргумент имеет большую длину}

\begin{lstlisting}
     (
     defun
         longer_then
         (list_1 list_2)
         (> (length list_1) (length list_2))
     )

     (LONGER_THEN '(1 2 3) '(2 3))       ; T
     (LONGER_THEN '(1 2 3) '(2 3))       ; NIL
     (LONGER_THEN '(2 3 1) '(1 2 3))     ; NIL
\end{lstlisting}


\section{Каковы результаты вычисления следующих выражений?}

\begin{lstlisting}
     (cons 3 (list 5 6))                     ; (3 5 6)
     (cons 3 '(list 5 6))                    ; (3 list 5 6)
     (list 3 'from 9 'lives (- 9 3))         ; (3 from 9 lives 6)
     (+ (length for 2 two))(car '(21 22 23)) ; error
     (cdr '(cons is short for ans))          ; (is short for ans)
     (car (list one two))                    ; error
     (car (list 'one 'two))                  ; one
\end{lstlisting}

\section{Дана функция. Какие будут результаты вычисления выражений}

\begin{lstlisting}
     (
     defun
        f1
        (x)
        (list (second x)(first x))
     )
\end{lstlisting}

\begin{lstlisting}
     (f1 (one two))                ; error
     (f1 (last one two))           ; error
     (f1 free)                     ; error
     (f1 one 'two)                 ; error
\end{lstlisting}

\section{Написать функцию, которая переводит температуру в системе Фаренгейта в температуру по Цельсию}

\begin{lstlisting}
     (
     defun
        f_to_c
        (t)
        (* (/ 5 9) (- t 32.0))
     )
     (f_to_c 451) ; 232.777
\end{lstlisting}

\begin{lstlisting}
     (
     defun
        c_to_f
        (t)
        (+ (* (/ 9 5) t) 32.0)
     )
     (c_to_f 232.777) ; 450.999
\end{lstlisting}

\section{Что получится при вычислении каждого из выражений}

\begin{lstlisting}
     (list 'cons t NIL)                     ; (cons t ())
     (eval (list 'cons t NIL))              ; (t)
     (eval (eval (list 'cons t NIL)))       ; error
     (eval NIL)                             ; NIL
     (apply #cons ''(t nill)))              ; error
     (list 'eval NIL)                       ; (eval NIL)
     (eval (list 'eval NIL))                ; NIL
\end{lstlisting}
