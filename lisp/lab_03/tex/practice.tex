\chapter{Практические задания}

\section{Написать функцию, которая принимает целое число и возвращает первое четное число, больше или равное аргумента}

\begin{lstlisting}
     (
     defun
        f1
        (x)
        (if (= (mod x 2) 0) x (+ x 1))
     )

     (f1 4)     ; 4
     (f1 5)     ; 6
\end{lstlisting}

\section{Написать функцию, которая принимает число и возвращает число того же знака, но с модулем на 1 больше модуля аргумента}

\begin{lstlisting}
     (
     defun
        f2
        (x)
        (if (< x 0) (+ x -1) (+ x 1))
    )

    (f2 2.5)    ; 3.5
    (f2 -1.0)   ; -2
\end{lstlisting}

\section{Написать функцию, которая принимает два числа и возвращает
список из этих чисел, расположенный по возрастанию}

\newpage 

\begin{lstlisting}
     (
     defun
        f3
        (x y)
        (if (< x y) (list x y) (list y x))
    )

    (f3 2 1.2)      ; (1.2 2)
    (f3 5 6.0)      ; (5 6)
\end{lstlisting}

\section{Написать функцию, которая принимает три числа и возвращает Т только тогда, когда первое число расположено между вторым и третьим}

\begin{lstlisting}
     (
     defun
        f4
        (x1 x2 x3)
        (and (< x2 x1) (< x1 x3) T)
    )

    (f4 1 2 3)      ; NIL
    (f4 2 1 3)      ; T
\end{lstlisting}

\section{Каков результат вычисления следующих выражений?}

\begin{lstlisting}
     (and 'fee 'fie 'foe)                   ; foe
     (or nil 'fie 'foe)                     ; fie
     (and (equal 'abc 'abc) 'yes)           ; yes
     (or 'fee 'fie 'foe)                    ; fee
     (and nil 'fie 'foe)                    ; nil
     (or (equal 'abc 'abc) 'yes)            ; T
\end{lstlisting}

\section{Написать предикат, который принимает два числа-аргумента и возвращает
Т, если первое число не меньше второго}

\begin{lstlisting}
     (
     defun 
        my_geq
        (x1 x2)
        (>= x1 x2)
     )

     (my_geq 1 2)       ; nil
     (my_geq 2 1)       ; T
     (my_geq 2 2)       ; T
\end{lstlisting}

\section{Какой из следующих двух вариантов предиката ошибочен и почему?}

\begin{lstlisting}
     ; NUMBERP -- проверяет, является ли значение аргумента числом
     ; PLUSP   -- проверяет, является ли число положительным
     
     (
     defun 
        pred1 
        (x) 
        (and (numberp x) (plusp x))
    )

    (
    defun 
        pred2 
        (x)
        (and (plusp x)(numberp x))
    )
\end{lstlisting}

Первый вариант предиката корректен --- он сначала проверят, что аргумент является числом, после, что число положительно.

Ошибочным является второй предикат, так как ему нельзя подать на вход строку (в этом случае выводится сообщение об ошибке).
      
\begin{lstlisting}
    (pred1 1)       ; T
    (pred1 -1)      ; NIL
    (pred1 "a")     ; NIL

    (pred2 1)       ; T
    (pred2 -1)      ; NIL
    (pred2 "a")     ; ERROR: Not a number!
\end{lstlisting}


\section{Решить задачу 4, используя для ее решения конструкции:
только IF, только COND, только AND/OR}

Написать функцию, которая принимает три числа и возвращает Т только тогда, когда первое число расположено между вторым и третьим.

\begin{lstlisting}
     (
     defun
        use_if
        (x1 x2 x3)
        (if (< x2 x1) (< x1 x3) NIL)
     )

     (use_if 1 2 3)     ; NIL
     (use_if 2 1 3)     ; T
     (use_if 2 1 2)     ; NIL
\end{lstlisting}

\begin{lstlisting}
     (
     defun
        use_cond
        (x1 x2 x3)
        (cond ((< x2 x1) (< x1 x3)) )
     )

     (use_cond 1 2 3)    ; NIL
     (use_cond 2 1 3)    ; T
     (use_cond 2 1 2)    ; NIL
\end{lstlisting}

\begin{lstlisting}
     (
     defun
        use_and
        (x1 x2 x3)
        (and (< x2 x1) (< x1 x3))
    )

    (use_and 1 2 3)      ; NIL
    (use_and 2 1 3)      ; T
    (use_and 2 1 2)      ; NIL
\end{lstlisting}


\section{Переписать функцию how-alike, приведенную в лекции и использующую COND, используя только конструкции IF, AND/OR}

Функция вернет 
\begin{itemize}
    \item the-same, если числа равны;
    \item both-odd, если числа не равны и оба четны;
    \item both-even, если числа не равны и оба нечетны;
    \item difference, иначе.
\end{itemize}

\begin{lstlisting}
     (
     defun
        how_alike 
        (x y) 
        (
        cond 
            ( (or (= x y) (equal x y)) 'the_same )
            ( (and (oddp x) (oddp y)) 'both_odd )
            ( (and (evenp x) (evenp y)) 'both_even )
            ( T 'difference) 
        )
     )
\end{lstlisting}

\subsection*{IF}

\begin{lstlisting}
     (
     defun
        how_alike_if
        (x y)
        (
        if 
            (or (= x y) (equal x y))            ; условие1
            'the_same                           ; если выполнено
            (
            if
                (and (oddp x) (oddp y))         ; условие2
                'both_odd                       ; если выполнено
                (
                if 
                    (and (evenp x) (evenp y))   ; условие 3
                    'both_even                  ; если выполнено
                    'difference                 ; иначе
                )
            )
        )
    )
\end{lstlisting}

\begin{lstlisting}
     (how_alike_if 1 2)     ; DIFFERENCE
     (how_alike_if 2 2)     ; THE_SAME
     (how_alike_if 2 4)     ; BOTH_EVEN
     (how_alike_if 3 1)     ; BOTH_ODD
\end{lstlisting}

\newpage

\subsection*{AND/OR}

\begin{lstlisting}
     (
     defun
        how_alike_and
        (x y)
        (
        or 
            (
            and 
                (or (= x y) (equal x y))
                'the_same  
            )
            (
            and 
                (and (oddp x) (oddp y))
                'both_odd   
            )
            (
            and 
                (and (evenp x) (evenp y))
                'both_even
            )
            'difference  
        )
     )
\end{lstlisting}

\begin{lstlisting}
     (how_alike_and 1 2)     ; DIFFERENCE
     (how_alike_and 2 2)     ; THE_SAME
     (how_alike_and 2 4)     ; BOTH_EVEN
     (how_alike_and 3 1)     ; BOTH_ODD
\end{lstlisting}