\chapter{Практические задания}

\section{Чем принципиально отличаются функции cons, list, append?}

Отличия:
\begin{itemize}
    \item cons является базисной, list и append -- нет;
    \item  list и append принимают произвольное количество аргументов (причем аргументами append могут быть только списки), cons -- фиксированное (два);
    \item  cons создает точечную пару или список (в зависимости от второго аргумента), list и append -- список;
    \item cons и list создают новые списковые ячейки (все), а append имеет общие списковые ячейки с последним списком.
    \item list и append определяются с помощью cons.
\end{itemize}

\begin{lstlisting}
     (setf lst1 '(a b c)) 
     (setf lst2 '(d e))

     (cons lst1 lst2)            ; ((a b c) d e)  
     (list lst1 lst2)            ; ((a b c) (d e))
     (append lst1 lst2)          ; (a b c d e)  
\end{lstlisting}

\section{Каковы результаты вычисления следующих выражений, и почему?}

Функция reverse переворачивает свой список-аргумент, т.е. меняет порядок ; его элементов  верхнего уровня на противоположный.

\begin{lstlisting}
     (reverse '(a b c))             ; (c b a)
     (reverse '(a b (c (d))))       ; ((c (d)) b a)
     (reverse '(a))                 ; (a)
     (reverse ())                   ; ()
     (reverse '((a b c)))           ; (a b c)
\end{lstlisting}

Функция last возвращает последнюю cons-ячейку в списке.

\begin{lstlisting}
     (last '(a b c))                ; (c)
     (last '(a))                    ; (a)
     (last '((a b c)))              ; ((a b c))
     (last '(a b (c)))              ; ((c))
     (last ())                      ; ()
\end{lstlisting}

\section{Написать, по крайней мере, два варианта функции, которая возвращает последний элемент своего списка-аргумента}

\begin{lstlisting}
     (
        defun
        my_last1
        (l)
        (cond ((null l) nil)
              ((null (cdr l)) (car l))
              (t (my_last (cdr l)))
        )
     )

     (my_last1 a)               ; (3)
\end{lstlisting}

\begin{lstlisting}
     (
        defun
        my_last2
        (l)
        (cons (car (reverse l) nil))
     )

     (my_last2 a)               ; (3)
\end{lstlisting}

\section{Написать, по крайней мере, два варианта функции, которая возвращает свой список аргумент без последнего элемента}

\begin{lstlisting}
     (
        defun
        without_last1
        (l)
        (
        if 
            (null (cdr l))
            nil
            (cons (car l) (without_last1 (cdr l)))
        )
     )

     (WITHOUT_LAST1 a)          ; (1 2)
\end{lstlisting}

\begin{lstlisting}
     (
        defun
        without_last2
        (l)
        (reverse (cdr (reverse l)))
     )

     (WITHOUT_LAST2 a)         ; (1 2)
\end{lstlisting}

\section{Напишите функцию swap-first-last, которая переставляет в списке-аргументе первый и последний элементы}

\begin{lstlisting}
     (
        defun 
        swap-first-last 
        (a) 
        (
            append 
            (cons(car (reverse (cdr a))) nil)
            (reverse (cdr (reverse (cdr a))))
            (cons (car a) nil)
        )
     )

     (SWAP-FIRST-LAST '(1 2 3 4 5))         ; (5 2 3 4 1)
     (SWAP-FIRST-LAST '((1 2) 3 4 (4) 5))   ; (5 3 4 (4) (1 2))
\end{lstlisting}

\section{Написать простой вариант игры в кости, в котором бросаются две
правильные кости. Если сумма выпавших очков равна 7 или 11 —
выигрыш, если выпало (1,1) или (6,6) — игрок имеет право снова
бросить кости, во всех остальных случаях ход переходит ко второму
игроку, но запоминается сумма выпавших очков. Если второй игрок не
выигрывает абсолютно, то выигрывает тот игрок, у которого больше
очков. Результат игры и значения выпавших костей выводить на экран с
помощью функции print}

\begin{lstlisting}
    (
        defun 
        roll_dice 
        () 
        (+ (random 6) 1)
    )
    
    (
        defun 
        check_continue_game 
        (result) 
        (not (or (= result 7) (= result 11)))
    )
    
    (
        defun 
        make_a_move 
        (player_i) 
        (
            let (
                (dice1 (roll_dice)) 
                (dice2 (roll_dice))
            )
    	(
            if (
                and 
                    (print (list 'Игрок player_i 'бросает 'кости)) 
                    (= dice1 dice2) 
                    (or (= dice1 1) (= dice1 6))
                )
    		(
                and 
    			(print (list 'Выпало  dice1 dice2  'Повторный 'бросок)) 
    			(make_a_move player_i)
    		)
    		(
                and 
    			(print (list 'Выпало dice1 dice2))
    			(+ dice1 dice2)
    		)
    	)
        )
    )
    (
        defun 
        compare_results 
        (res1 res2) 
        (
            if 
            (check_continue_game res2)
    	  (
            and
    		  (print (list 'Сравнение 'по 'очкам))
    		  (print (list 'Игрок 1 'набрал res1))
    		  (print (list 'Игрок 2 'набрал res2))
    		  (cond 
    			((< res1 res2) (and (print (list 'Игрок 2 'выиграл 'по 'очкам)) 2))
    			((> res1 res2) (and (print (list 'Игрок 1 'выиграл 'по 'очкам)) 1))
    			((and (print '(Ничья)) 0))
    		  )
    	  )	
    	  (
            and 
                (print (list 'Игрок 2 'набрал res2 'очков 'и 'выиграл 'абсолютно)) 2
            )
        )
    )
    
    (
        defun 
        play_game 
        () 
        (
            let (
                (res1 (make_a_move 1))
            )
            (if (check_continue_game res1)
                (compare_results res1 (make_a_move 2))
                (and (print (list 'Игрок 1 'набрал res1 'очков 'и 'выиграл 'абсолютно)) 1)
            )
        )
    )


    ; (ИГРОК 1 БРОСАЕТ КОСТИ) 
    ; (ВЫПАЛО 1 5) 
    ; (ИГРОК 2 БРОСАЕТ КОСТИ) 
    ; (ВЫПАЛО 6 1) 
    ; (ИГРОК 2 НАБРАЛ 7 ОЧКОВ И ВЫИГРАЛ АБСОЛЮТНО) 
\end{lstlisting}

\section{Написать функцию, которая по своему списку-аргументу lst определяет является ли он палиндромом (то есть равны ли lst и (reverse lst))}

\begin{lstlisting}
    (
        defun
        is_palindrom
        (l)
        (equal l (reverse l))
    )

    (IS_PALINDROM '(1 2 3 4 5))     ; NIL
    (IS_PALINDROM '(1 2 3 2 1))     ; T
    (IS_PALINDROM '((1 2) 3 (2 1))) ; NIL
    (IS_PALINDROM '((1 2) 3 (1 2))) ; T
\end{lstlisting}


\section{Напишите свои необходимые функции, которые обрабатывают таблицу из 4-х точечных пар: (страна . столица), и возвращают по стране - столицу, а по столице — страну}

\begin{lstlisting}
    (
    defvar country_table
    '(
       (Россия . Москва)
       (Англия . Лондон)
       (Италия . Рим)
       (Испания . Мадрид)
    )
    )
    
    (
        defun
        get_country_by_capital
        (capital table)
        (
            cond 
                ((null table) "Не найдено")
                ((equal (cdr (car table)) capital) (car (car table)))
                (t (get_country_by_capital capital (cdr table)))
        )
    )
    
    (
        defun
        get_capital_by_country
        (country table)
        (
            cond 
                ((null table) "Не найдено")
                ((equal (car (car table)) country) (cdr (car table)))
                (t (get_capital_by_country country (cdr table)))
        )
    )
    
    (
        defun
        get_by_value
        (value table)
        (
            if
            (equal "Не найдено" (get_country_by_capital value table))
            (get_capital_by_country value table)
            (get_country_by_capital value table)
        )
    )
    
    (print (get_country_by_capital 'ФФФФ country_table))
    (print (get_country_by_capital 'МОСКВА country_table))
    (print (get_country_by_capital 'ЛОНДОН country_table))
    (print (get_country_by_capital 'МАДРИД country_table))
    
    (print (get_capital_by_country 'ФФФФ country_table))
    (print (get_capital_by_country 'РОССИЯ country_table))
    (print (get_capital_by_country 'АНГЛИЯ country_table))
    (print (get_capital_by_country 'ИСПАНИЯ country_table))
    
    (print (get_by_value 'FFF country_table))
    (print (get_by_value 'Рим country_table))
    (print (get_by_value 'Италия country_table))
\end{lstlisting}

\section{Напишите функцию, которая умножает на заданное число-аргумент
первый числовой элемент списка из заданного 3-х элементного списка аргумента, когда a) все элементы списка --- числа, b) элементы списка -- любые объекты.}

\begin{lstlisting}
    (
        defun
        simple_option
        (x l)
        (
            cons (* (car l) x) (cdr l)
        )
    )
    
    (print (simple_option 5 '(1 2 3 4 5)))                 ; (5 2 3 4 5) 

    
    (
        defun
        difficult_option
        (x l)
        (
            cond 
            ((null l) "Невозможно умножить (нет числовых значений)")
            ((numberp (car l)) (cons (* (car l) x) (cdr l)))
            (t (cons (car l) (difficult_option x (cdr l))))
        )
    )
    
    (difficult_option 5 '(1 2 3 4 5))              ; (5 2 3 4 5) 
    (difficult_option 5 '((1) 2 3 4 5))            ; ((1) 10 3 4 5) 
    (difficult_option 5 '("1" (2 3 4) 5))          ; ("1" (2 3 4) 25) 
    (difficult_option 5 '("1" (2 3 4) (5)))        ; "Невозможно умножить (нет числовых значений)"

\end{lstlisting}
