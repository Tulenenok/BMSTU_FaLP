\chapter{Теоретические вопросы}

\textbf{Функционал (функция высшего порядка)} --- функция, аргументом или результатом которой является другая функция.

\section{APPLY}

Функционал apply является обычной функцией с двумя вычисляемыми аргументами, обращение к ней имеет вид 
\begin{lstlisting}
    (apply f l)         ;   f -- функциональный аргумент 
                        ;   l -- список фактических параметров
\end{lstlisting}

Значение функционала --- результат применения f к этим фактическим параметрам. Примеры:

\begin{lstlisting}
    (apply (lambda (x y) (* x y)) '(9 8))       ; 72

    (defun f1 (x y) (* x y))
    (apply 'f1 '(9 8))                          ; 72
\end{lstlisting}

\section{FUNCALL}

Функционал funcall --- особая функция с вычисляемыми аргументами, обращение к ней
\begin{lstlisting}
    (funcall f e1 .. en)          ; f -- функциональный аргумент 
                                  ; n >= 0
\end{lstlisting}

Действие аналогично apply, отличие в том, что аргументы передаются не в виде списка, а по отдельности. Примеры:

\begin{lstlisting}
    (funcall (lambda (x y) (* x y)) 9 8)        ; 72

    (defun f1 (x y) (* x y))
    (funcall 'f1 9 8)                           ; 72
\end{lstlisting}

\section{MAPCAR}

Значение функции mapcar вычисляется путем применения функции fn к последовательным элементам xi списка, являющегося вторым аргументом mapcar.

\begin{lstlisting}
    (mapcar f (x1 x2 ... xn))    ; f -- функциональный аргумент 
                                 ; n >= 0

    (mapcar f (x1 y1 z1...) (x2 y2 z2...) (xn yn zn...)))
\end{lstlisting}

В качестве значения функционала возвращается список, построенный из результатов вызовов функционального аргумента mapcar.

\begin{lstlisting}
    (mapcar (lambda (x) (list x)) '(a b c))     ; ((A) (B) (C)) 
    (mapcar 'f2 '(1 2 3))                       ; (3 6 9)
    (mapcar 'f1 '(1 2) '(3 4) )                 ; (3 8)         !!!
\end{lstlisting}

\section{MAPLIST}

Функционал maplist принимает два аргумента. Значение первого аргумента должно быть функцией списком. Второй аргумент - список. Функция, задаваемая первым аргументом, должна принимать на вход список. Выполнение функционала заключается в том, что функция, заданная первым аргументом, последовательно применяется к значению второго аргумента, к этому списку без первого элемента, без первых двух элементов и т.д. до исчерпания списка. Результаты вызова функции объединяются в список, который функционал вернет в качестве значения. 

Примеры:
\begin{lstlisting}
    (maplist (lambda (x) x) '(a b c))       ; ((A B C) (B C) (C))
    (maplist 'reverse '(a b c))             ; ((C B A) (C B) (C)) 
\end{lstlisting}