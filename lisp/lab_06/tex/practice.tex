\chapter{Практические задания}

\section{Написать хвостовую рекурсивную функцию my-reverse, которая развернет верхний уровень своего списка-аргумента lst}

\begin{lstlisting}
    ; С помощью append
    (
        defun my_reverse1 (lst)
            (if
                (null lst) 
                nil
                (append (my_reverse1 (cdr lst)) (list (car lst)))
            )
    )
    
    (print (my_reverse1 '()))               ; NIL
    (print (my_reverse1 '(1 2 3 4 5)))      ; (5 4 3 2 1)
    (print (my_reverse1 '(1 (2 3) 4)))      ; (4 (2 3) 1) 
    
    
    ; Без append
    (
        defun my_reverse2 (lst &optional (result ()))
            (if
                (null lst)
                result
                (my_reverse2 (cdr lst) (cons (car lst) result))
            )
    )
    
    (print (my_reverse2 '()))               ; NIL
    (print (my_reverse2 '(1 2 3 4 5)))      ; (5 4 3 2 1)
    (print (my_reverse2 '(1 (2 3) 4)))      ; (4 (2 3) 1) 
\end{lstlisting}

\section{Написать функцию, которая возвращает первый элемент списка - аргумента, который сам является непустым списком}

\begin{lstlisting}
    (
        defun return_first_list (lst)
            (cond
                (
                    (null lst)
                    nil
                )
                (
                    (and (not (atom (car lst))) (not (null (car lst))))  
                    (car lst)
                )
                (
                    t 
                    (return_first_list (cdr lst))
                )
            )
    )
    
    (print (return_first_list '(1 ())))                 ; nil
    (print (return_first_list '(1 2 3 4 5)))            ; nil
    (print (return_first_list '((1 2) 3 4)))            ; (1 2)
    (print (return_first_list '(1 2 (3 4))))            ; (3 4)
    (print (return_first_list '(1 (2) (3 4))))          ; (2)
    (print (return_first_list '(1 () (3 4))))           ; (3 4)
\end{lstlisting}

\section{Напишите рекурсивную функцию, которая умножает на заданное число-аргумент все числа из заданного списка-аргумента, когда все элементы списка --- числа, элементы списка -- любые объекты}

\begin{lstlisting}
    ; a
    (
        defun a_multiply_all (lst x)
            (if
                (null lst)
                nil
                (cons (* x (car lst)) (a_multiply_all (cdr lst) x))
            )
    )
    
    (print (a_multiply_all '() 2))                    ; nil
    (print (a_multiply_all '(1 2 3 4 5) 2))           ; (2 4 6 8 10) 
    
    
    ; b
    (
        defun b_multiply_all (lst x)
            (cond
                ((null lst) nil)
                ((not (numberp (car lst))) (cons (car lst) (b_multiply_all (cdr lst) x)) )
                (t (cons (* x (car lst)) (b_multiply_all (cdr lst) x)))
            )
    )
    
    (print (b_multiply_all '() 2))                    ; nil
    (print (b_multiply_all '(1 2 3 4 5) 2))           ; (2 4 6 8 10) 
    (print (b_multiply_all '(1 (2) 3 4 5) 2))         ; (2 (2) 6 8 10) 
    (print (b_multiply_all '(1 (2) 3 "1" 5) 2))       ; (2 (2) 6 "1" 10) 
\end{lstlisting}

\section{Напишите функцию, select-between, которая из списка-аргумента, содержащего только числа, выбирает только те, которые расположены между двумя указанными границами-аргументами и возвращает их в виде списка}

\begin{lstlisting}
    (
        defun get_numbers (lst start end result flag)
            (cond
                ((null lst) nil)
                ((= start (car lst)) (cons (car lst) (get_numbers (cdr lst) start end result 1))) 
                ((= end (car lst)) (cons (car lst) (get_numbers (cdr lst) start end result 0))) 
                ((= 1 flag) (cons (car lst) (get_numbers (cdr lst) start end result 1)))
                ((= 0 flag) (get_numbers (cdr lst) start end result 0))
            )
    )
    
    (
        defun select_between (lst start end)
            (get_numbers lst start end () 0)
    )
    
    ; Сортировка
    (
        defun put_elem_in_result (x result)
            (cond
                ((null result) (cons x result))
                ((<= x (car result)) (cons x result))
                (t (cons (car result) (put_elem_in_result x (cdr result))))
            )
    )
    
    (
        defun my_sort (lst &optional (result ()))
            (if
                (null lst)
                result
                (my_sort (cdr lst) (put_elem_in_result (car lst) result))
            )
    )
    
    ; (print (my_sort '(5 4 3 2 1)))       ; (1 2 3 4 5)
    ; (print (my_sort '(5 14 3 12 1)))     ; (1 3 5 12 14) 
    
    (select_between '(1 2 3 4 5 6) 1 6)               ; (1 2 3 4 5 6)
    (select_between '(1 2 3 4 5 6) 2 4)               ; (2 3 4) 
    (select_between '(10 2 5 4 1 9) 2 1)              ; (2 5 4 1)  
    
    (my_sort (select_between '(10 2 5 4 1 9) 2 1))    ; (2 5 4 1)  
    (my_sort (select_between '(10 23 -5 4 1 9) -5 9)) ; (-5 1 4 9)   
\end{lstlisting}

\section{Написать рекурсивную версию (с именем rec-add) вычисления суммы чисел заданного списка одноуровнего смешанного, структурированного}

\begin{lstlisting}
    (
        defun a_rec_add (lst &optional (result 0))
            (cond
                ((null lst) result)
                ((numberp (car lst)) (a_rec_add (cdr lst) (+ result (car lst))))
                (t (a_rec_add (cdr lst) result))
            )
    )
    
    (print (a_rec_add '()))               ; 0
    (print (a_rec_add '(1 2 3 4 5)))      ; 15
    (print (a_rec_add '(1 (2) 3 4 5)))    ; 13
    
    
    (
        defun b_rec_add (lst &optional (result 0))
            (cond
                ((null lst) result)
                ((numberp (car lst)) (b_rec_add (cdr lst) (+ result (car lst))))
                ((listp (car lst)) (+ (b_rec_add (car lst) 0) (b_rec_add (cdr lst) result)))
                (t (a_rec_add (cdr lst) result))
            )
    )
    
    (print (b_rec_add '()))                    ; 0
    (print (b_rec_add '(1 2 3 4 5)))           ; 15
    (print (b_rec_add '(1 (2) 3 4 5)))         ; 16
    (print (b_rec_add '(1 (4) 3 4 5)))         ; 18
    (print (b_rec_add '(1 (4) 3 "4" 5)))       ; 14
    (print (b_rec_add '(1 (4 3) 3)))           ; 11
    (print (b_rec_add '(1 (4 (3 5)) 3)))       ; 16
\end{lstlisting}