\chapter{Теоретические вопросы}

\textbf{Рекурсия} --- это ссылка на определяемый объект во время его определения. Т.к. в Lisp используются рекурсивно определенные структуры, то рекурсия --- это естественный принцип обработки таких структур. Существуют типы рекурсивных функций: хвостовая,
дополняемая, множественная, взаимная рекурсия и рекурсия более высокого порядка.

\section{Хвостовая рекурсия}

Частный случай рекурсии, при котором любой рекурсивный вызов является последней операцией перед возвратом из функции.
    
\begin{lstlisting}
    ; Пример хвостовой рекурсии для нахождения 
    ; суммы элементов конечного списка
    
    (defun sum_tail (lst acc)
      (if (null lst)
          acc
          (sum-tail (cdr lst) (+ acc (car lst)))))
    
    (print (sum_tail '(1 2 3 4 5) 0))               ; 15 
\end{lstlisting}

\section{Множественная рекурсия}

Рекурсия, содержащая несколько самоссылок.

\section{Рекурсия более высокого порядка}

Рекурсия более высокого порядка --- это когда функция вызывает саму себя, но передает в качестве аргумента другую функцию.

\section{Взаимная рекурсия}

Вид рекурсии, когда два математических или программных объекта, таких как функции или типы данных, определяются в терминах друг друга.

\newpage

\begin{lstlisting}
    ; Пример взаимной рекурсии для определения четности числа
    (
        defun is-even (n)
            (if 
                (= n 0)
                t
                (if 
                    (= n 1)
                    nil
                    (is-odd (- n 1))
                )
            )
    )
    
    (
        defun is-odd (n)
            (if 
                (= n 0)
                nil
                (if 
                    (= n 1)
                    t
                    (is-even (- n 1))
                )
            )
    )
    
    (print (is-odd 1))      ; T
    (print (is-odd 2))      ; NIL
    (print (is-even 2))     ; T
\end{lstlisting}

\section{Дополняемая рекурсия}

Форма рекурсии, при которой вызов рекурсивной функции происходит после выполнения последней операции в теле функции. Обычно это означает, что результат предыдущего вызова рекурсивной функции используется для вычисления результата текущего вызова.

\begin{lstlisting}
    ; Пример дополняемое рекурсии для определения факториала числа
    (
        defun factorial (n &optional (acc 1))
            (if 
                (<= n 1)
                acc
                (factorial (- n 1) (* n acc))
            )
    )

    (print (factorial 5))           ; 120 
\end{lstlisting}
